\include{includes/common_start}
\include{includes/tutBlatt_methods}
\tutnr{4}

\section{CYK}
\subsection{CYK-Algorithmus}

\begin{frame}
\frametitle{CYK Abschluss}
\begin{center}
\begin{tabular}{lll}
$S'\rightarrow S\mid \epsilon$&${\only<8-9,16>{\color{red}}D} \rightarrow {\only<16>{\color{red}}DD}\mid {\only<8-9>{\color{red}}d}$&${\only<6-7>{\color{red}}C} \rightarrow {\only<6-7>{\color{red}}c}$\\
${\only<13,37,69,80,85,92>{\color{red}}S} \rightarrow {\only<69,80,85,92>{\color{red}}XD}\mid {\only<13,37>{\color{red}}YC}$&${\only<13,37>{\color{red}}M}\rightarrow {\only<13,37>{\color{red}}YC}$&${\only<2-3,10,48,61>{\color{red}}X} \rightarrow {\only<48,61>{\color{red}}AM} \mid {\only<10>{\color{red}}AA} \mid {\only<2-3>{\color{red}}a}$\\
${\only<2-3,10>{\color{red}}A} \rightarrow {\only<10>{\color{red}}AA}\mid {\only<2-3>{\color{red}}a}$&${\only<4-5>{\color{red}}B} \rightarrow {\only<4-5>{\color{red}}b}$&${\only<4-5,21>{\color{red}}Y} \rightarrow {\only<21>{\color{red}}BM} \mid {\only<4-5>{\color{red}}b}$\\
\end{tabular}
\end{center}

\begin{center}
\begin{tabular}{|*{9}{C{20 pt}|}}
\hline
&{\only<2>{\color{red}}a}&{\only<3>{\color{red}}a}&{\only<4>{\color{red}}b}&{\only<5>{\color{red}}b}&{\only<6>{\color{red}}c}&{\only<7>{\color{red}}c}&{\only<8>{\color{red}}d}&{\only<9>{\color{red}}d}\\
\hline
{\only<2>{\color{red}}a}&\only<2>{\greenCell}\only<10,17,29,44,60,75,87>{\RedCell}\only<18,30-31>{\redCell}\only<2->{A,X}&\only<10>{\greenCell}\only<18,30,45,61,76,88>{\RedCell}\only<31>{\redCell}\only<10->{A,X}&\only<17-18>{\greenCell}\only<31,46,62,77,89>{\RedCell}&\only<29-31>{\greenCell}\only<47,63,78,90>{\RedCell}&\only<44-47>{\greenCell}\only<64,79,91>{\RedCell}&\only<60-64>{\greenCell}\only<80,92>{\RedCell}\only<61->{X}&\only<75-80>{\greenCell}\only<93>{\RedCell}\only<80->{S}&\only<87-94>{\greenCell}\only<92->{S}\\
\hline
{\only<3>{\color{red}}a}&\blackCell&\only<3>{\greenCell}\only<10-11,19,32,48,65,81>{\RedCell}\only<17-18,20,29-31,33-34>{\redCell}\only<3->{A,X}&\only<11>{\greenCell}\only<29,31,34>{\redCell}\only<17,20,33,49,66,82>{\RedCell}&\only<19-20>{\greenCell}\only<29,34,50,67,83>{\RedCell}&\only<32-34>{\greenCell}\only<44,51,68,84>{\RedCell}&\only<48-51>{\greenCell}\only<60,69,85>{\RedCell}\only<48->{X}&\only<65-69>{\greenCell}\only<75,86>{\RedCell}\only<69->{S}&\only<81-86>{\greenCell}\only<87>{\RedCell}\only<85->{S}\\
\hline
{\only<4>{\color{red}}b}&\blackCell&\blackCell&\only<4>{\greenCell}\only<11-12>{\RedCell}\only<17,19-20,22,29-34,36-37>{\redCell}\only<18,21,35,52,70>{\RedCell}\only<4->{B,Y}&\only<12>{\greenCell}\only<19,22,30,36,53,71>{\RedCell}\only<29,32,34,37>{\redCell}&\only<21-22>{\greenCell}\only<32,37,45,54,72>{\RedCell}\only<21->{Y}&\only<35-37>{\greenCell}\only<48,55,61,73>{\RedCell}\only<37->{S,M}&\only<52-55>{\greenCell}\only<65,74,76>{\RedCell}&\only<70-74>{\greenCell}\only<81,88>{\RedCell}\\
\hline
{\only<5>{\color{red}}b}&\blackCell&\blackCell&\blackCell&\only<5>{\greenCell}\only<12-13,20,23,31,38,56>{\RedCell}\only<19,21-22,24,29-30,32-37,39-40>{\redCell}\only<5->{B,Y}&\only<13>{\greenCell}\only<21,24,33,39,46,57>{\RedCell}\only<32,35,37,40>{\redCell}\only<13->{S,M}&\only<23-24>{\greenCell}\only<35,40,49,58,62>{\RedCell}&\only<38-40>{\greenCell}\only<52,59,66,77>{\RedCell}&\only<56-59>{\greenCell}\only<70,82,89>{\RedCell}\\
\hline
{\only<6>{\color{red}}c}&\blackCell&\blackCell&\blackCell&\blackCell&\only<6>{\greenCell}\only<13-14,22,25,34,41,47>{\RedCell}\only<21,23-24,26,32-33,35-40,42-43>{\redCell}\only<6->{C}&\only<14>{\greenCell}\only<23,26,36,42,50,63>{\RedCell}\only<35,38,40,43>{\redCell}&\only<25-26>{\greenCell}\only<38,43,53,67,78>{\RedCell}&\only<41-43>{\greenCell}\only<56,71,83,90>{\RedCell}\\
\hline
{\only<7>{\color{red}}c}&\blackCell&\blackCell&\blackCell&\blackCell&\blackCell&\only<7>{\greenCell}\only<14-15,24,27,37,51,64>{\RedCell}\only<23,25-26,28,35-36,38-43>{\redCell}\only<7->{C}&\only<15>{\greenCell}\only<25,28,39,54,68,79>{\RedCell}\only<38,41,43>{\redCell}&\only<27-28>{\greenCell}\only<41,57,72,84,91>{\RedCell}\\
\hline
{\only<8>{\color{red}}d}&\blackCell&\blackCell&\blackCell&\blackCell&\blackCell&\blackCell&\only<8>{\greenCell}\only<15-16,26,40,55,69,80>{\RedCell}\only<25,27-28,38-39,41-43>{\redCell}\only<8->{D}&\only<16>{\greenCell}\only<41>{\redCell}\only<27,42,58,73,85,92>{\RedCell}\only<16->{D}\\
\hline
{\only<9>{\color{red}}d}&\blackCell&\blackCell&\blackCell&\blackCell&\blackCell&\blackCell&\blackCell&\only<9>{\greenCell}\only<16,28,43,59,74,86,93>{\RedCell}\only<27,41-42>{\redCell}\only<9->{D}\\
\hline
\multicolumn{1}{c}{}\\
\end{tabular}

\only<94>{Sind die Wörter $abbcc$ und $abcdd$ in der Sprache der Grammatik?}
\end{center}
\end{frame}

\section{Kellerautomaten}
\subsection{Kellerautomaten}
\begin{frame}
	\frametitle{Definition Kellerautomaten}
	Ein (nichtdeterministischer) \textbf{Kellerautomat} (NPDA bzw PDA, Pushdown Automaton) besteht aus $(Q, \Sigma, \Gamma, q_0,\delta, F)$, wobei
	\begin{itemize}
		\item $Q$ endliche Zustandsmenge
		\item $\Sigma$ endliches Eingabealphabet
		\item $\Gamma$ endliches Stack-Alphabet
		\item $q_0 \in Q$ Anfangszustand
		\item $\delta : Q \times ( \Sigma \cup \{\epsilon\}) \times \Gamma \rightarrow 2^{Q \times \Gamma^*}$
		\begin{itemize}
			\item $\delta(q, a, Z) \subseteq \{(q,\gamma) : q \in Q, \gamma \in \Gamma^*\}$
			\item $\delta(q, \epsilon, Z) \subseteq \{(q,\gamma) : q \in Q, \gamma \in \Gamma^*\}$
		\end{itemize}
		\item $F \subseteq Q$ Menge der akzeptierenden Endzustände, $F=\emptyset$ ist möglich.
		
		\vspace{-4cm}\raggedleft{\includegraphics[width=0.5\textwidth]{images/PDA.png}}
	\end{itemize}
\end{frame}

\begin{frame}
\frametitle{Zu Kellerautomaten}
\begin{itemize}
\item Akzeptieren nach Eingabeende, wenn \begin{itemize}
	\item der Stack leer ist \emph{oder}
	\item der Automat in einen akzeptierenden Zustand kommt.
\end{itemize}
\item Sind im Allgemeinen nichtdeterministisch
\item Man kann Endzustände auch aus der Definition weglassen und alternativ verlangen, dass der Automat genau bei leerem Keller akzeptiert.
\item Man kann sogar alle Zustände bis auf einen weglassen und alles in die Kellerbelegung kodieren
\end{itemize}
\end{frame}

\begin{frame}
	\frametitle{Beispiel}
	$M = (Q, \Sigma, \Gamma, q_0, \delta, F)$
	\begin{itemize}
		\item $Q = \{q_0, q_1, q_2\}$
		\item $\Sigma = \{a,b\}$
		\item $\Gamma = \{\#,X\}$
		\item $F = \{q_2\}$
	\end{itemize}
	\begin{figure}
		\begin{tikzpicture}[node distance=2.5cm,shorten >=1pt,auto]
			\node[state,initial]   (q_0)                {$q_0$};
			\node[state]           (q_1) [right of=q_0] {$q_1$};
			\node[state]           (q_2) [right of=q_1] {$q_2$};
			\node[state,accepting] (q_3) [right of=q_2] {$q_3$};
			\path[->]	
			(q_0) 	edge 			node {$(\epsilon,\epsilon,\#)$}				(q_1)	
			(q_1) 	edge 			node {$(b,X,\epsilon)$}				(q_2)
			edge [loop above]	node {${(a,X,XX)} \atop {(a,\#,X\#)}$}	 	()
			(q_2)	edge			node {$(\epsilon,\#,\epsilon)$}			(q_3)
			edge [loop above]	node {$(b,X,\epsilon)$}				();
		\end{tikzpicture}
	\end{figure}
	\begin{itemize}
		\item Welche Sprache akzeptiert dieser Automat?
	\end{itemize}
\end{frame}

\subsection{Alte Aufgabe 3}
\begin{frame}
	\frametitle{Alte Aufgabe 3}
	Gegeben sei folgende Sprache f"ur das Alphabet $\Sigma = \{a,b,c\}$:
	\begin{multline*}
		\mathcal{L} = \{w_1w_2 \in \Sigma^* \; | \; w_1 \in \{a,b\}^*,w_2 \in \{b,c\}^*,\\
		\#_a w_1 + \#_b w_1 = \#_b w_2 + \#_c w_2\}
	\end{multline*}
	Hier gibt $\#_x w$ die H"aufigkeit des Vorkommens eines Zeichens $x \in \Sigma$ in
	einem Wort $w \in \Sigma^*$ an.
	\begin{enumerate}
		\item Zeigen Sie, dass $\mathcal{L}$ nicht regul"ar ist!
		\item Geben Sie eine Chomsky-2-Grammatik an, die genau die Sprache $\mathcal{L}$
		erzeugt!
		\item Geben Sie einen Kellerautomaten $\mathcal{M}$ an, der genau die Sprache
		$\mathcal{L}$ erkennt! Zeichnen Sie den\\
		Zustands"ubergangsgraphen f"ur $\mathcal{M}$!
	\end{enumerate}
\end{frame}

\section{Pumping Lemma für kontextfreie Sprachen}
\subsection{Pumping Lemma für kontextfreie Sprachen}
\begin{frame}
	\frametitle{Pumping-Lemma für kontextfreie Sprachen}
	\begin{exampleblock}{Lemma}
		Für jede kontextfreie Sprache $L$ gibt es eine Konstante $n \in \mathbb{N}$,
		so dass sich jedes Wort $z \in L$ mit $|z| \geq n$ so als
		$$ z = uvwxy $$
		schreiben lässt, dass
		\begin{itemize}
			\item $|vx| \geq 1$,
			\item $|vwx| \leq n$ und
			\item für alle $i \geq 0$ das Wort $uv^iwx^iy \in L$ ist.
		\end{itemize}
	\end{exampleblock}
\end{frame}

\begin{frame}
	\frametitle{Beweisidee}
	\begin{itemize}
		\item Jeder Knoten im Ableitungsbaum (wie wir ihn in CYK sehen) steht für ein Nichtterminalsymbol
		\item Ab einer gewissen Höhe des Baumes (bzw. Länge des Wortes) muss ein Nichtterminal im Baum mehrmals in einer Reihe vorkommen
		\item Man kann also aus einem Nichtterminalsymbol dasselbe Symbol wieder ableiten
		\item Da das Wort durch jede Ableitung (außer zu Terminalsymbolen) länger wird, gibt es eine "`Schleife"' beim Ableiten
		\item Diese Schleife kann man also "`pumpen"', also beliebig oft (oder auch gar nicht) durchlaufen
	\end{itemize}
\end{frame}

\begin{frame}
	\frametitle{Beweisidee}
	\begin{figure}[H]
		\centering
		\includegraphics[scale=0.41]{images/pumping}
	\end{figure}
\end{frame}

\frame{
	\frametitle{Pumping Lemma Formalia (kontextfrei)}
	Behauptung: L ist nicht kontextfrei. ~\\
	Beweis:
	\begin{enumerate}
	\item[] Nehme an L sei kontextfrei.% Sei n wie im Pumping Lemma gefordert.
	\item[] Sei n beliebig aber fest.
	%\item Pumping Lemma: $\exists n \in N$, so dass jedes Wort $z \in L$ mit $|z| \ge n$ eine Zerlegung z = uvwxy besitzt mit $|vx|\ge 1$ und $|vwx|\le n$, so dass $uv^iwx^iy \in L$  für $\forall i\in \N_0$
	\item[] Wähle z=\underline{\hspace{3cm}} $\in L$ mit $|z| \ge n$
	%\item Zeige, dass für alle Zerlegungen von z, die den Regeln des Pumping Lemmas genügen, ein i existiert, sodass $uv^iwx^iy\not\in L$
	\item[] Beh.: $\forall u,v,w,x,y: uvwxy=z$ mit $|vx|\ge 1$ und $|vwx|\le n$, $\exists i \in N$, so dass $uv^iwx^iy \not\in L$.
	\item[] Bew.:\underline{\hspace{8cm}}
	\item[] Widerspruch zum Pumping Lemma $\Rightarrow$ L ist nicht kontextfrei.
	\end{enumerate}
}

\begin{frame}
	\frametitle{Beispiel}
	Zeige, dass die Sprache
	\[L=\{\omega \omega|\omega \in \{0,1\}^*\}\]
	nicht kontextfrei ist.
\end{frame}

\subsection{Aufgabe 1}
\begin{frame}
	\frametitle{Aufgabe 1}
	\begin{enumerate}
		\item Geben Sie f"ur die Sprache $\mathcal{L} = \{a^nb^nc^n \; | \; n \in
		\mathbb{N}\}$ eine Grammatik des h"ochstm"oglichen Chomsky-Typs an!
		\item Zeigen Sie, dass die Sprache $\mathcal{L}' = \{a^{2^n} \; | \; n \in
		\mathbb{N}\}$ nicht kontextfrei ist!
	\end{enumerate}
\end{frame}

\section{Schluss}
\subsection{Schluss}

\begin{frame}
\frametitle{Bis zum nächsten Mal!}
\begin{center}
  \includegraphics[width=1 \textheight]{images/password_strength.png}
\end{center}
\end{frame}

\include{includes/common_end}