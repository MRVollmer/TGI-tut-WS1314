\include{includes/common_start}
\include{includes/tutBlatt_methods}
\include{amsmath}
\tutnr{7}

\section{Altlasten}
\subsection{Ch1}
\begin{frame}
	\frametitle{CH1 -- Not even my final Form}
	Kontextfreie Grammatiken werden in dieser Vorlesung durch Produktionsmengen der folgenden Form charakterisiert:
\begin{itemize}
	\item $P$ = Menge der Produktionen mit Form $v \rightarrow w$
	\begin{itemize}
		\item $v \in V^{+}$
		\item $w \in ((V \setminus \{S\}) \bigcup T)^{+}$
		\item $|v| \leq |w|$ oder $S \rightarrow \epsilon$
	\end{itemize}
\end{itemize}
Grammatiken dieser Form lassen sich alle in Grammatiken dieser Form umwandeln (siehe Vorlesung bzw. Skript):
\begin{itemize}
	\item $P$ = Menge der Produktionen mit Form $\alpha A \beta \rightarrow \alpha B \beta$ oder $S \rightarrow A$ oder $A \rightarrow a$
	\begin{itemize}
		\item $A \in V$
		\item $\alpha, \beta, B \in V^{*}, B \neq \epsilon$
		\item $a in T$
	\end{itemize}
\end{itemize}
Das ist wichtig weil:
\begin{itemize}
	\item Die untere Form ist geläufiger als Definition von kontextfreien Grammatiken
	\item Dazu konstruierte TM bei der unteren Form u.U. einfacher sind
\end{itemize}
\end{frame}

\begin{frame}
	\frametitle{Chompsky Abschlüsse}
	\begin{tabular}{|*{5}{c|}}
	\hline
	&Ch3&Ch2&Ch1&Ch0\\
	\hline
	Name&regulär&kontextfrei&kontextsensitiv&rekursiv aufzählbar\\
	\hline
	Entscheidbar&$\checkmark$&$\checkmark$&$\checkmark$&semi\\
	\hline
	\glqq $\cdot$\grqq-Abschluss&$\checkmark$&$\checkmark$&$\checkmark$&$\checkmark$\\
	\hline
	\glqq $-$\grqq-Abschluss&$\checkmark$&$\times$&$\checkmark$&$\times$\\
	\hline
	\glqq $\cup$\grqq-Abschluss&$\checkmark$&$\checkmark$&$\checkmark$&$\checkmark$\\
	\hline
	\glqq $\cap$\grqq-Abschluss&$\checkmark$&$\times$&$\checkmark$&$\checkmark$\\
	\hline
	\glqq $*$\grqq-Abschluss&$\checkmark$&$\checkmark$&$\checkmark$&$\checkmark$\\
	\hline
	\end{tabular}
	\begin{itemize}
		\item \glqq $-$\grqq-Abschluss = Abgeschlossenheit unter Komplementbildung
		\item Semientscheidbarkeit = $\exists$TM, die genau alle Wörter der Sprache akzeptiert, aber Wörter außerhalb der Sprache können Endlosschleifen erzeugen
		\item nicht entscheidbar = nichtentscheidbar = unentscheidbar = Kann nicht für jedes Wort sagen ob es in der Sprache liegt oder nicht, kann aber semi-entscheidbar sein
	\end{itemize}
\end{frame}

\subsection{Reduktion}

\begin{frame}
\frametitle{Reduktion}
Aufgabe: Ist ein gegebenes $Problem$ A $attribut$?~\\~\\
\begin{itemize}
\item Nehme an, A ist $attribut$
\item Suche ein geeignetes $Problem$ B, das bekanntermaßen (laut Vorlesung) $nicht$ $attribut$ ist
\item Zeige: Wenn A $attribut$ ist, dann wäre B auch $attribut$
\begin{itemize}
	\item Transformiere \textbf{alle} Instanzen von B zu Instanzen von A, wobei diese Transformation $attribut$ \textbf{nicht beeinflussen} darf.
\end{itemize}
\item Widerspruch!
\end{itemize}
\end{frame}

\begin{frame}
\vspace{-1 cm}
Ist die Sprache $L=\{\langle M \rangle \mid \text{TM M hat mind. einen nicht erreichbaren Zustand}\}$ entscheidbar?
\begin{itemize}
	\item Annahme: $L$ entscheidbar ($\Leftrightarrow \overline{L}$ entscheidbar)
	\item Bekannt: Das Halteproblem ist nicht entscheidbar
	\item Transformation f von (allen) Instanzen $\in$ Halt zu Instanzen von $\overline{L}$~\\ $f:(\langle M \rangle, w) \rightarrow \langle M' \rangle$
	\item Konstruiere $M'$: $M'$ hat folgende Funktionsweise:
	\begin{enumerate}
		\item Leere das Band
		\item Schreibe w auf das Band
		\item Simuliere $M$
		\item Gehe in einen zusätzlichen Zustand $q_s$
	\end{enumerate}
	\item Folgerung:
	\begin{itemize}
		\item $\langle M' \rangle = f((\langle M \rangle, w)) \in \overline{L}$
		\item $\Leftrightarrow$ $M'$ hat keinen nicht erreichbaren Zustand
		\item $\Leftrightarrow$ $M'$ geht in Zustand $q_s$
		\item $\Leftrightarrow$ $M$ hält bei Eingabe w
		\item $\Leftrightarrow (\langle M \rangle, w) \in HALT$
	\end{itemize}
	\item Also: $L$ entscheidbar $\Rightarrow \overline{L}$ entscheidbar $\Rightarrow$ HALT entscheidbar $\lightning$
\end{itemize}
\end{frame}

\begin{frame}
	\frametitle{Mehr Übungen}
	Ist die Sprache
	\begin{enumerate}[$L_1 =$]
		\item $\{\langle M \rangle \mid$ TM $M$ akzeptiert keine Eingabe$\}$
		\item $\{\langle M \rangle \mid$ TM $M$ akzeptiert die Eingabe $\langle M \rangle$ nicht$\}$
		\item $\{\langle M \rangle \mid$ TM $M$ ist minimal$\}$
		\begin{itemize}
			\item[] d.h. es gibt keine funktionsäquivalente Turingmaschine $N$ mit $|\langle N \rangle | < |\langle M \rangle |$
			\item Beweis siehe Skript und/oder Tutorium 8
		\end{itemize}
	\end{enumerate}
	entscheidbar?
\end{frame}

\subsection{Aufgabe B6 A3}
\begin{frame}
	\frametitle{Aufgabe B6 A3 rekursiv aufzählbare Mengen}
	Welche der folgenden Mengen sind rekursiv aufz"ahlbar? \\
	Beweisen Sie Ihre Aussage!
	\begin{enumerate}
		\item $M_2 := \{r \in \mathbb{R} \; | \; 0<r<1\}$
	\end{enumerate}
\end{frame}

\section{Rekursionstheorem}
\subsection{Rekursionstheorem erklären}
\begin{frame}
\begin{block}{Das Rekursionstheorem 1.Form}
Existiert eine TM M, die die Funktion t: $\Sigma^* \times \Sigma^* \rightarrow \Sigma^*$ berechnet, dann existiert eine TM R die t($\langle R\rangle$,w) berechnet, wobei w die Eingabe ist.
\end{block}
Dieses Theorem ist nicht nur auf Turingmaschinen beschränkt, sondern kann auch auf jede beliebige turingvollständige Codierungsform (wie z.B. Programmiersprachen) ausgedehnt werden.
\begin{block}{Das Rekursionstheorem 2.Form}
Für jede berechenbare Funktion f: $\Sigma^*\rightarrow\Sigma^*$ existiert eine TM F und eine TM G, wobei F und G die gleiche Funktion berechnen und $f(\langle F\rangle)=\langle G\rangle$.\\
\end{block}
\end{frame}

\subsection{SELF-Maschine / Quines erklären}
\begin{frame}
\frametitle{SELF-Maschine}
Eine SELF-Maschine (auch Quine genannt) ist eine Turingmaschine, die ihre eigene Gödelnummer ausgibt und dann hält. Sie realisiert demnach die Funktion $t(\langle SELF\rangle,w)=\langle SELF\rangle$.\\
Eine mögliche Art eine solche TM zu erstellen ist folgender:
\begin{itemize}
\item Man zerlegt die Turingmaschine in zwei Teile A und B.
\item Teil A löscht die Eingabe und schreibt die Gödelnummer von Teil B aufs Band.
\item Teil B liest die neue Eingabe w (seine eigene Gödelnummer) ein, schreibt die Gödelnummer der Turingmaschine aufs Band die bei beliebiger Eingabe das Wort w ausgibt, hängt daran w an und hält.
\end{itemize}
\end{frame}

\begin{frame}
\frametitle{\glqq Übung\grqq}
Beweisen Sie, dass es eine G"odelnummer $n = \langle\mathcal{M}\rangle \in
\mathbb{N}_0$ zu einer Turingmaschine $\mathcal{M}$ gibt, die die Funktion\\
$f_n(x) = (n+x)^2$ f"ur alle $x \in \mathbb{N}_0$ berechnet!
\end{frame}

\section{Formale Logik}
\subsection{Wiederholung einiger Begriffe}
\begin{frame}
	\frametitle{Wiederholung einiger Begriffe}
	\begin{itemize}
		\item Quantoren
		\begin{itemize}
			\item Existenzquantor $\exists x$: \\ Aussage muss für mindestens ein x aus dem Universum gelten.
			\item Allquantor $\forall x$: Aussage muss für alle x aus dem Universum gelten.
			\item Vorsicht bei Schachtelung von Quantoren: \\ $\forall x \exists y: x = y$ ist etwas völlig anderes als $\exists y \forall x: x = y$.
		\end{itemize}
		\item Ein Universum ist die Menge über der man eine Aussage betrachtet.
		\item Eine Relation drückt aus, dass zwei Objekte zueinander in Beziehung stehen.
		\begin{itemize}
			\item Sei $R$ die Gleichheit, dann gilt $R(x, y) \Leftrightarrow x = y$.
		\end{itemize}
		\item Eine Theorie ist eine Menge $Th(U, R)$ induziert über dem Tupel $(U, R)$ mit einem Universum $U$ und einer Relation $R$. \\ 
		Eine Formel $\phi$ ist Element einer Theorie, falls sie in Bezug auf $U$ bzw. $R$ wahr ist.
		\begin{itemize}
			\item Sei $\phi = \forall x \exists y: R_1(x,y)$. Dann gilt $\phi \in Th(\mathbb{Z}, >)$ aber $\phi \notin Th(\mathbb{N}, >)$.
		\end{itemize}
	\end{itemize}
\end{frame}

\subsection{B7 A2}
\begin{frame}
	\frametitle{Weitere Aufgaben: B7 A2}
	Geben Sie f"ur folgendende Formeln an ob diese in den besagten Theorien liegen
	\begin{enumerate}
		\item Ist $\phi_1 = \forall x \exists y \forall z: x + y = z$ in $\text{Th}(\mathbb{N,+})$?
		\item Ist $\phi_2 = \forall x \exists y \forall z \exists w: (x + z = w ) \wedge (x + y = w)$  in $\text{Th}(\mathbb{N},+)$?
		\item Ist $\phi_3 = \forall x \forall y \forall z \forall w \forall v \exists s: \neg(x + w = y) \vee \neg(y + v = z) \vee (x + s = z)$ in $\text{Th}(\mathbb{N},+)$?
		\item Sei $\text{Th}(\mathbb{N},<)$ die Theorie der nat"urlichen Zahlen mit der Relation "`echt kleiner"'. Zeigen Sie: $\text{Th}(\mathbb{N},<)$ ist entscheidbar.
	\end{enumerate}
\end{frame}
\subsection{B7 A3}
\begin{frame}
	\frametitle{Weitere Aufgaben: B7 A3}
	Geben Sie Modelle f"ur die folgenden pr"adikatenlogischen Formeln an! Geben Sie dazu
	jeweils ein Universum $\mathcal{U}$\\
	und eine Interpretation der Relationszeichen $R_i$ an!
	\begin{enumerate}
		\item $\phi_1 =$ \hspace*{0.2cm} $\forall \; x \; (R_1(x,x))$ \hspace*{5.3cm}[K1.1]\\
		\hspace*{0.85cm} $\wedge \forall \; x,y \; (R_1(x,y) \leftrightarrow R_1(y,x))$
		\hspace*{3.2cm} [K1.2]\\
		\hspace*{0.85cm} $\wedge \forall \; x,y,z \; ((R_1(x,y) \wedge R_1(y,z)) \rightarrow
		R_1(x,z))$ \hspace*{1cm} [K1.3]
		\item $\phi_2 =$ \hspace*{0.2cm} $\phi_1$\\
		\hspace*{0.85cm} $\wedge \forall \; x \; (R_1(x,x) \rightarrow \neg R_2(x,x))$
		\hspace*{3.25cm} [K2.1]\\
		\hspace*{0.85cm} $\wedge \forall \; x,y \; (\neg R_1(x,y) \rightarrow (R_2(x,y)
		\oplus R_2(y,x)))$ \hspace*{1cm} [K2.2]\\
		\hspace*{0.85cm} $\wedge \forall \; x,y,z \; ((R_2(x,y) \wedge R_2(y,z)) \rightarrow
		R_2(x,z))$ \hspace*{1cm}  [K2.3]\\
		\hspace*{0.85cm} $\wedge \forall \; x \; \exists \; y \, (R_2(x,y))$ \hspace*{4.8cm}
		[K2.4]
	\end{enumerate}
\end{frame}
%\subsection{B6 A4}
%\begin{frame}
%	\frametitle{Weitere Aufgaben: B6 A4}
%	Sei $A \subseteq \mathbb{N}_0$ eine entscheidbare Menge. Zeigen Sie, dass
%	$ B := \{x+2y^2+17+11^x \; | \; x,y \in A\}$ entscheidbar ist!
%\end{frame}

\section{Schluss}
\subsection{Schluss}
\begin{frame}
\frametitle{Bis zum nächsten Mal!}
\begin{center}
  \includegraphics[width=1 \textheight]{images/xkcd_981.png}
\end{center}
\end{frame}

\include{includes/common_end}
